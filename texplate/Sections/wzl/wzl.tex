\newpage
\section{wzl}
	\subsection{素数和积}
		\begin{flushleft}
			If p is a prime and $p^2 \leq v$, we have $S(v,p) = S(v,p-1) - (S(v/p,p-1) - S(p-1,p-1))$.
			\linebreak Similarly, $F(v,p) = F(v,p-1) / p^{S(v/p,p-1)-S(p-1,p-1)} / (F(v/p,p-1) / S(p-1,p-1))$.
			\linebreak s表示,2-v 用 p 筛完还剩下的数的和
			\linebreak f表示2-v用p筛完 除了m的倍数的剩下的数的积
		\end{flushleft}
		\lstinputlisting{Sections/wzl/sushuheji.cpp}
	\subsection{BM线性递推}
		\lstinputlisting{Sections/wzl/BM线性递推.cpp}
	\subsection{FFT}
		\lstinputlisting{Sections/wzl/fft.cpp}
	\subsection{NTT}
		\lstinputlisting{Sections/wzl/ntt.cpp}
	\subsection{Gauss}
		\lstinputlisting{Sections/wzl/gauss.cpp}
	\subsection{mu,phi,prime筛}
		\lstinputlisting{Sections/wzl/mu,phi,prime筛.cpp}
	\subsection{polysum}
		\lstinputlisting{Sections/wzl/polysum.cpp}
	\subsection{杜教筛}
		\begin{flushleft}
			$g(1)S(n) = \sum_{i = 1}^{n}h(i) - \sum_{d = 2}^{n}g(d) \times S(\lfloor\frac{n}{d}\rfloor) $
		\end{flushleft}
		\lstinputlisting{Sections/wzl/杜教筛.cpp}
	\subsection{类欧几里得}
		\lstinputlisting{Sections/wzl/类欧几里得.cpp}
	\subsection{min 25}
		\begin{flushleft}
			定义$\sigma(n)=n$的因子数
			\linebreak $\sum_{i=1}^n\sigma(i^k)  \%  2^{64}  $
		\end{flushleft}
		\lstinputlisting{Sections/wzl/min_25_1.cpp}
		\begin{flushleft}
			$求phi( i^2 ) 1e9 \% 1e9+7 $
		\end{flushleft}
		\lstinputlisting{Sections/wzl/min_25_2.cpp}
		\begin{flushleft}
			定义f(n)=n的最小质因子
			\linebreak $\sum_{i=1}^{n}f(i)   \% 2^{64}$
		\end{flushleft}
		\lstinputlisting{Sections/wzl/min_25_3.cpp}
	\subsection{NTT 三模数}
		\lstinputlisting{Sections/wzl/NTT三模数.cpp}
	\subsection{原根}
		\lstinputlisting{Sections/wzl/原根.cpp}