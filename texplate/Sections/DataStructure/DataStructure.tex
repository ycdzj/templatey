\newpage
\section{Data Structure}
	\subsection{Basic ZKW}
		\begin{flushleft}
			zkw神树,堆式存储结构,根节点为1,两倍空间。
			\linebreak 使用方法:init --> [] --> build --> query/update
			\linebreak op需要满足:结合律、交换律。
		\end{flushleft}
		\lstinputlisting{Sections/DataStructure/basic_zkw.cpp}
	\newpage
	\subsection{Segment Tree}
		\begin{flushleft}
			这里仅给出线段树的一种设计模式,线段树的适用范围十分广泛,应当根据不同的要求,设计出不同的线段树。
			\begin{itemize}
				\item 节点的值:存储什么样的信息才能快速完成各项操作。
				\item pushdown:把节点的信息等价转化给孩子(将更改下推)。
				\item merge:把两个节点合并(根据下层的值构造上层的值)。
			\end{itemize}
			下面给出一个例子,实现了区间更改/区间查询最小值的功能。
		\end{flushleft}
		\lstinputlisting{Sections/DataStructure/SegmentTree.cpp}
	\newpage
	\subsection{Persistent Segment Tree}
		\begin{flushleft}
			\begin{itemize}
				\item 初学Persistent DS, MIT的公开课还没看明白。
				\item pointer machine
				\item version
				\item partial / full persistent, thinking / method / analysis
			\end{itemize}
			下面是一棵支持单点修改、区间求和的持久化线段树。
		\end{flushleft}
		\lstinputlisting{Sections/DataStructure/psgt.cpp}
	\newpage
	\subsection{Binary Indexed Tree}
		\begin{flushleft}
			需要注意的地方:
			\begin{itemize}
				\item 下标从1开始,使用一倍存储空间
				\item 树状数组的第i个位置存放原数组以$a_{i}$为最后一个元素的lowbit(i)个元素的和
				\item lowbit可用来快速查找二进制表示中最低位的1
			\end{itemize}
		\end{flushleft}
		\lstinputlisting{Sections/DataStructure/BIT.cpp}
	\newpage
	\subsection{Treap}
		\begin{flushleft}
			写得OOP了一些,可以适应更多种情况,用了STL的allocator,性能没有下降太多。与一般log算法相比,常数大约要乘以2。
			\linebreak 初始时仅需创建空指针(在这里即是一棵空树),大部分接口传入的都是指针的引用。
			\linebreak 如果常数还是过大,可以考虑不使用构造函数
			\linebreak 实现的接口:size, insert, remove, select, lower\_bound, upper\_bound, contain
		\end{flushleft}
		\lstinputlisting{Sections/DataStructure/Treap.cpp}
	\newpage
	\subsection{Splay Tree}
		\lstinputlisting{Sections/DataStructure/SplayTree.cpp}
