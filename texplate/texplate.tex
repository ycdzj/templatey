\documentclass[twoside, UTF8]{article}

%\usepackage{showframe}

\usepackage{geometry}
\geometry{left = 1.5cm, right = 1.5cm, top = 2.0cm, bottom = 1.0cm}

\usepackage{listings}
\usepackage{color}
\usepackage{courier}
\renewcommand\lstlistingname{} % Change language of section name
\lstset{ % General setup for the package
	language=C++,
	frame=tb,
	basicstyle=\small\ttfamily,
	keywordstyle=\color{blue},
	numbers=left,
%	numbersep=5pt,
	numberstyle=\tiny,
	tabsize=4,
	breaklines=true,
	postbreak=\mbox{\textcolor{red}{$\hookrightarrow$}},
	showstringspaces=false	
}

\usepackage{fancyhdr}
\pagestyle{fancy}
\fancyhf{}
\fancyhead[RO, LE]{\leftmark}
\fancyhead[LO, RE]{\thepage}

\usepackage{afterpage}
\newcommand\blankpage{ \null\thispagestyle{empty}\addtocounter{page}{-1}\newpage }

\usepackage{ctex}
%\usepackage{CJK}

\title{Standard Code Library}
\author{}


\begin{document}
%	\begin{CJK}{UTF8}{gkai}
	\pagenumbering{gobble}
	\maketitle
	\afterpage{\blankpage}

	\newpage
	\tableofcontents

	\newpage
	\pagenumbering{arabic}
	\section{String algorithm}
		\subsection{Kmp}
			\lstinputlisting{../src/String/Kmp.cpp}
		\subsection{Longest Palindrome}
			\lstinputlisting{../src/String/Manacher.cpp}
		\subsection{Extend Kmp}
			\lstinputlisting{../src/String/ExKmp.cpp}
		\subsection{AC Automaton}
			\lstinputlisting{../src/String/ACAutomaton.cpp}
		\subsection{Suffix Arrary}
			\lstinputlisting{../src/String/SuffixArray.cpp}
	\newpage
	\section{Data Structure}
		\subsection{Basic ZKW}
			\begin{flushleft}
				经典的zkw神树,堆式存储结构,根节点为1,两倍空间。
				\linebreak 使用方法:init --> [] --> build --> query/update
				\linebreak op需要满足:结合律、交换律。
			\end{flushleft}
			\lstinputlisting{../src/DataStructure/basic_zkw.cpp}
		\subsection{Segment Tree}
			\begin{flushleft}
				这里仅给出线段树的一种设计模式,线段树的适用范围十分广泛,应当根据不同的要求,设计出不同的线段树。
				\begin{itemize}
					\item 节点的值:存储什么样的信息才能快速完成各项操作。
					\item pushdown:把节点的信息等价转化给孩子(将更改下推)。
					\item merge:把两个节点合并(根据下层的值构造上层的值)。
				\end{itemize}
				下面给出一个例子,实现了区间更改/区间查询最小值的功能。
			\end{flushleft}
			\lstinputlisting{../src/DataStructure/SegmentTree.cpp}
		\subsection{Binary Indexed Tree}
			\begin{flushleft}
				需要注意的地方:
				\begin{itemize}
					\item 下标从1开始,使用一倍存储空间
					\item 树状数组的第i个位置存放原数组以$a_{i}$为最后一个元素的lowbit(i)个元素的和
					\item lowbit可用来快速查找二进制表示中最低位的1
				\end{itemize}
			\end{flushleft}
			\lstinputlisting{../src/DataStructure/BIT.cpp}
		\subsection{Treap}
			\begin{flushleft}
				写得OOP了一些,可以适应更多种情况,用了STL的allocator,性能没有下降太多。与一般log算法相比,常数大约要乘以2。
				\linebreak 初始时仅需创建空指针(在这里即是一棵空树),大部分接口传入的都是指针的引用。
				\linebreak 如果常数还是过大,可以考虑不使用构造函数
				\linebreak 实现的接口:size, insert, remove, select, lower\_bound, upper\_bound, contain
			\end{flushleft}
			\lstinputlisting{../src/DataStructure/Treap.cpp}
		\subsection{Splay Tree}
			\lstinputlisting{../src/DataStructure/SplayTree.cpp}
	\newpage
	\section{Graph}
		\subsection{Maximum Flow}
			\begin{flushleft}
				最大流的一些知识点:
				\begin{itemize}
					\item 流网络的切割(净流量、容量)
					\item 残量网络的定义、增广路对残量网络的增广
					\item 最大流最小割定理
					\item FORD-FULKERSON方法、EK算法
				\end{itemize}
				最大流的dinic算法,每次先在残量网络中bfs构建层次图(level),再dfs寻找增广路。
				\linebreak 需要注意的地方:
				\begin{itemize}
					\item dinic\_dfs里的当前边优化
					\item 每次add\_edge的时候都一次加两条,方便找到反向边。(果然不能读书读死了啊。。只要不用邻接矩阵就可以加入反向边的)
				\end{itemize}
			\end{flushleft}
			\lstinputlisting{../src/Graph/dinic.cpp}
		\newpage
		\subsection{Min Cost Flow}
			\begin{flushleft}
				把EK算法的BFS改成了SPFA。
			\end{flushleft}
			\lstinputlisting{../src/Graph/minCostFlow.cpp}
		\newpage
		\subsection{Matching and Covers}
			\begin{flushleft}
				定义:
				\begin{itemize}
					\item A \textbf{vertex cover} of a graph is a set of vertices such that each edge of the graph is incident to at least one vertex of the set.
					\item An \textbf{edge cover} of a graph is a set of edges such that every vertex of the graph is incident to at least one edge of the set.
					\item A \textbf{matching} or \textbf{independent edge set} of a graph is a set of edges without common vertices.
					\item An \textbf{independent set} of a graph is a set $S$ of vertices such that for every two vertices in $S$, there is no edge connecting the two.
					\linebreak 以上都是定义在任何类型的{\tiny (无向?)}图上的?
					\item A \textbf{path cover} of a directed graph is a set of directed paths such that every vertex of the graph belongs to \textbf{exactly} one path.
				\end{itemize}
			\end{flushleft}
		
			\begin{flushleft}
				定理:
				\begin{itemize}
					\item 对于全部二分图, $|$Minimum Vertex Cover$|$ $=$ $|$Maximum Matching$|$
					\item 对于全部{\tiny (无向?)}图$G=(V,E)$, $|$Minimum Edge Cover$|$ $=$ $|$V$|$ - $|$Maximum Matching$|$
					\item 对于全部DAG,按一般方法把一个点拆成两个,令新二分图的最大匹配为x,$|$原图的最小路径覆盖$|$=$|V| - x$
					\item 对于全部不带有向环的传递闭包,$|$最大独立集$|$ $=$ $|$最小路径覆盖$|$ (为啥啊???)
					\item HALL's theory
				\end{itemize}
			\end{flushleft}
			
			\begin{flushleft}
				求\textbf{二分图}最大匹配的匈牙利算法\textbf{(只适用于二分图)},单向边和双向边都适用。
				\linebreak 每次dfs寻找一条未匹配/已匹配边交错出现的路径(与寻找一条网络流模型中的增广路等价)。
			\end{flushleft}
			\lstinputlisting{../src/Graph/Maximum_Matching.cpp}
		\newpage
		\subsection{Cut and Bridge and BCC}
			\begin{flushleft}
				求\textbf{无向图}的割点、割边(桥)、点双连通分量的tarjan算法,使用了dfn和low。
				\linebreak bcc编号从1开始。
			\end{flushleft}
			\lstinputlisting{../src/Graph/cut_bridge.cpp}
			\begin{itemize}
				\item $u$ $is$ $a$ $cut$ $\Leftrightarrow \exists v, <u, v> \in T \wedge Min[v] == d[u]$
				\item $pre$ $is$ $a$ $bridge$ $\Leftrightarrow Min[u] == d[u] - 1 \wedge \forall <u, v> \in T, Min[v] == d[u]$
			\end{itemize}
		\newpage
		\subsection{Strongly Connected Components}
			\begin{flushleft}
				tarjan的求强连通分量算法,还是使用dfn和low,只用树边和后向边更新low,不管cross-edge。
				\linebreak u是其所在scc的第一个被搜到的点,当且仅当$low[u] == dfn[u]$
				\linebreak scc编号从1开始。
			\end{flushleft}
			\lstinputlisting{../src/Graph/scc.cpp}
	\newpage
	\section{2D Geometry}
		\subsection{Basic Operations}
			\begin{flushleft}
				计算几何中向量的基本操作,大致分为:加减、数乘(除)、点积(dot)、叉积(cross)、旋转、求长度、求单位向量,同时把点也当作向量。
				\linebreak 参数不一定要传引用,成员只有两个double也快不了多少。尽量不用构造函数,太慢了。。
			\end{flushleft}
			\lstinputlisting{"../src/Geometry/2D Basic.cpp"}
		\subsection{Circle}
			\begin{flushleft}
				和圆有关的一些操作
				\linebreak 求两圆的交点是训练指南上的板子,返回交点个数,结果放在res中
			\end{flushleft}
			\lstinputlisting{"../src/Geometry/Circle.cpp"}
	\newpage
	\section{Number Theory}
		\subsection{Prime Number}
			\begin{flushleft}
				素数筛,$a[i]$代表$i$的素因子分解式中最小的素数。
			\end{flushleft}
			\lstinputlisting{../src/Math/prime.cpp}
		\subsection{Extend Euclid}
			\lstinputlisting{../src/Math/euclid.cpp}
	\newpage
	\section{Other}
		\subsection{Checker}
			\lstinputlisting[language=sh]{../src/Other/checker.sh}
		\subsection{Fast IO}
			\lstinputlisting{../src/Other/FastIO.cpp}
%\end{CJK}
\end{document}
